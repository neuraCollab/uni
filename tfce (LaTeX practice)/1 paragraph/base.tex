\documentclass{article}


\usepackage[utf8]{inputenc} % Поддержка UTF-8
\usepackage[russian]{babel} % Поддержка русского языка
\usepackage{amsmath, amsfonts, amssymb} % Для математических выражений и символов
\usepackage{geometry} % Настройка полей страницы
\geometry{top=2cm, bottom=2cm, left=2.5cm, right=2.5cm}
\usepackage{enumitem} % Списки с кастомными настройками
\usepackage{hyperref} % Для гиперссылок

\title{Теория функции комплексной переменной}
\author{Шарабарин Михаил}
\date{26.01.2025}

\begin{document}

\maketitle
\tableofcontents
\newpage

\section{Вопрос 1. Определения и формы комплексных чисел и действия над ними}

Комплексные числа — это числа, представляемые в виде:
\[
    z = a + bi,
\]
где \(a\) — действительная часть, \(b\) — мнимая часть,
а \(i\) (\(i = \sqrt{-1}\)) — мнимая единица.

Основные представления комплексных чисел:
\begin{itemize}
    \item Алгебраическая форма: \(z = a + bi\),
    \item Тригонометрическая форма: \(z = r (\cos \varphi + i \sin \varphi)\),
    \item Экспоненциальная форма: \(z = r e^{i\varphi}\),
\end{itemize}
где \(r = |z|\) — модуль числа, \(\varphi = \arg z\) — аргумент числа.

\subsection{Действия с комплексными числами}
\begin{itemize}
    \item {\bfseries Сложение}: \((a + bi) + (c + di) = (a + c) + (b + d)i\),
    \item {\bfseries Умножение}: \((a + bi) \cdot (c + di) = (ac - bd) + (ad + bc)i\),
    \item {\bfseries Деление}:
          \[
              \frac{a + bi}{c + di} = \frac{(a + bi)(c - di)}{c^2 + d^2}.
          \]
\end{itemize}

Представления комплексных чисел:
\begin{itemize}
    \item \(\mathfrak{Re}(a + b\mathbf{i}) = a\)
    \item \(\mathfrak{Im}(a + b\mathbf{i}) = b\)
    \item \begin{equation*}
              \overline{\bar{z}}=z,
              \quad \overline{z_1z_2}=\bar{z}_1\bar{z}_2,
              \quad \overline{\frac{z_1}{z_2}}=\frac{\bar{z}_1}{\bar{z}_2}.
          \end{equation*}
\end{itemize}

\textbf{Комплексное число} - это число \(z\) вида \(a + b\mathbf{i}\), где a -
вещественная часть, а b - мнимая. Символ \(\mathbf{i^2}=-1\) называется
мнимой единицей. \\
\textbf{Сопряженное число} - \( \overline{z}\) называется сопряженным числом и
записывается как \(a - b\mathbf{i}\). Свойства сопряженны чисел:
\begin{equation*}
    \overline{z*z} = \overline{z} * \overline{z},
    \quad \overline{\overline{z}} = z,
    \quad \overline{\frac{z_1}{z_2}} = \frac{\overline{z}}{\overline{z}}
\end{equation*}


В полярных координатах точка \(m\) имеет коодинаты \(r , \varphi\), (мы рассматриваем)
комплексные числа как вектора + их нельзя сравнить). Иногда говорят что вектор
и комплексное число это тоже самое. Здесь расстояние от центра координат до
этой точки будет равно модулю вектора
\[ r = \overline{OM} = |z| = \sqrt{z*\overline{z}} = \sqrt{a^2 + b^2}\]
в это время \(\varphi\) является углом между вектором \(\overline{OM}\) и
вектором \(\overline{OX}\) (направлением оси \(X\)) и обозначается как
\(\varphi=\mbox{Arg }z \)

\subsection{Особый разговор об аргументе}
\quad Аргумент - бесконечен т.к. все его значения отдаляются от истенных (в нап
равлении от оси \(\overline{OX}\) в противоположном направлении до \(2\pi\)). \\
Аргумент определяется в виде формулы
\begin{eqnarray}
    \begin{cases}
        x = r\cos{\varphi} \\
        x = r\sin{\varphi} \\
    \end{cases}
\end{eqnarray}

С точностью до слагаемого
\(\mbox{Arg }z = \arg{z} + 2\pi k, \quad k=0,\pm1,\pm2,\ldots,.\) \\
Из всех главных значений особо выделяются \(-\pi < \arg{z} < \pi \).
При этом полезны формулы:
\begin{eqnarray}
    \begin{cases}
        \arctan{\frac{y}{x}},       & \text{если} x > 0          \\
        \arctan{\frac{y}{x}} + \pi, & \text{если} x < 0, y \ge 0 \\
        \arctan{\frac{y}{x}} - \pi, & \text{если} x > 0, y < 0
    \end{cases}
\end{eqnarray}
Условия сопряжения двух чисел \(z\) и \(\overline{z}\):
\[
    \arg{z} = -\arg{\overline{z}}, \quad |z| = |\overline{z}|
\]
Некоторые свойства модуля:
\[
    |z_1 + z_2| \le |z_1| + |z_2|, \quad |z_1 + z_2| \ge ||z_1| - |z_2||,
    \quad |z_1 - z_2| \ge ||z_1| - |z_2||
\]

\subsection{Тригонометрическая форма комплексного числа}
\[
    z = i(\cos{\varphi} + \mathbf{i}\sin{\varphi})
\] \\
Формула Эйлера:
\[
    e^{i\varphi} = \cos{\varphi} + \mathbf{i}\sin{\varphi}
\] \\
Показательная форма числа:
\[
    z = re^{i\varphi}
\]

\subsection{Умножение и деление комплексных чисел,
    записанных в тригонометрической или показательной формах}
Пусть даны два комплексных числа:
\[
    \begin{array}{l}
        z_1=r_1(\cos\varphi_1+\mathbf i \sin\varphi_1), \\
        z_2=r_2(\cos\varphi_2+\mathbf i \sin\varphi_2).
    \end{array}
\]

Выведем формулу произвеения:
\[
    z_1z_2=r_1r_2\big(\cos\varphi_1\cos\varphi_2-\sin\varphi_1\sin
\varphi_2+\mathbf i (\sin\varphi_1\cos\varphi_2+\cos\varphi_1\sin\varphi_2)\big) = r_1r_2\big(\cos(\varphi_1+
\varphi_2)+\mathbf i \sin(\varphi_1+\varphi_2)\big).
\]

Деление:

\[
    \frac{z_1}{z_2} = \frac{r_1(\cos\varphi_1 + i\sin\varphi_1)}{r_2(\cos\varphi_2 + i\sin\varphi_2)} = 
    \frac{r_1(\cos\varphi_1 + i\sin\varphi_1)(\cos\varphi_2 - i\sin\varphi_2)}{r_2(\cos\varphi_2 + i\sin\varphi_2)(\cos\varphi_2 - i\sin\varphi_2)} = 
\]
\[
    = \frac{r_1}{r_2} \left( \cos\varphi_1\cos\varphi_2 + \sin\varphi_1\sin\varphi_2 + \mathbf{i}(\sin\varphi_1\cos\varphi_2 - \cos\varphi_1\sin\varphi_2) \right) = 
\]
\[
    = \frac{r_1}{r_2} \left( \cos\varphi_1\cos\varphi_2 + \sin\varphi_1\sin\varphi_2 + \mathbf{i}(\sin\varphi_1\cos\varphi_2 - \cos\varphi_1\sin\varphi_2) \right) =
\]

\subsection{Возведение в степень и извлечение корня из комплексного числа}

\[
    z^n=(x+\mathbf i y)^n=\big(r(\cos\varphi+\mathbf i \sin\varphi)\big)^n=
    \left(re^{\mathbf i \varphi}\right)^n=r^ne^{\mathbf i n\varphi}=r^n(\cos n\varphi+\mathbf i \sin n\varphi).
\]
Формула Муавра:
\[
   \cos n\varphi+\mathbf i \sin n\varphi
\]

Нахождение корня
\[
    w_k=\sqrt[n]{r(\cos\varphi+\mathbf i \sin\varphi)}=\!\sqrt[n]{r}\left(\cos
\frac{\arg z+2\pi k}n+\mathbf i \sin\frac{\arg z+2\pi k}n\right).
\]



\section{Комплексные числа}

\subsection{Алгебраическая форма}
Комплексное число представляется в виде:
\[
z = x + iy
\]
Где $\operatorname{Re}(z) = x$, $\operatorname{Im}(z) = y$.

\subsection{Свойства сопряжения}
\[
\overline{z} = x - iy
\]
\[
\overline{z_1 + z_2} = \overline{z_1} + \overline{z_2}, \quad \overline{z_1 z_2} = \overline{z_1} \cdot \overline{z_2}
\]
\[
\overline{\left(\frac{z_1}{z_2}\right)} = \frac{\overline{z_1}}{\overline{z_2}}
\]

\subsection{Производные}
\[
\frac{d}{dz} (x - iy)
\]

\subsection{Для любых чисел}
\begin{enumerate}
    \item Коммутативность: \( a + b = b + a \)
    \item Ассоциативность: \( (a + b) + c = a + (b + c) \)
    \item Дистрибутивность: \( a (b + c) = ab + ac \)
\end{enumerate}

\subsection{Полярные координаты}
Комплексное число можно записать в полярных координатах $(r, \varphi)$:
\[
z = r (\cos \varphi + i \sin \varphi)
\]
\[
r = |z| = \sqrt{x^2 + y^2}
\]
\[
\varphi = \arg z
\]

\subsection{Формула Эйлера}
\[
e^{i \varphi} = \cos \varphi + i \sin \varphi
\]
\[
z = r e^{i\varphi}
\]

\subsection{Формула Муавра}
\[
z^n = r^n e^{i n \varphi} = r^n (\cos n \varphi + i \sin n \varphi)
\]

\section{Множества комплексных чисел}

\subsection{Определения}
\begin{itemize}
    \item Диск: \( |z - z_0| < R \)
    \item Круг: \( |z - z_0| \leq R \)
\end{itemize}

\subsection{Внутренность и граница}
Множество $D$:
\begin{itemize}
    \item Внутренние точки: все точки, принадлежащие $D$ и содержащие окрестность, полностью лежащую в $D$.
    \item Граница: множество всех граничных точек.
    \item Открытое множество: если все его точки являются внутренними.
    \item Связное множество: если любые две точки можно соединить ломаной, лежащей в $D$.
\end{itemize}




\end{document}
