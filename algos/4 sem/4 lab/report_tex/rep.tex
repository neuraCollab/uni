%%%%%%%%%%%%%%%%%%%%%%%%%%%%%%%%%%%%%%%%%%%%%%%%%%%%%%%%%%%%%%%
%
% Welcome to Overleaf --- just edit your LaTeX on the left,
% and we'll compile it for you on the right. If you open the
% 'Share' menu, you can invite other users to edit at the same
% time. See www.overleaf.com/learn for more info. Enjoy!
%
%%%%%%%%%%%%%%%%%%%%%%%%%%%%%%%%%%%%%%%%%%%%%%%%%%%%%%%%%%%%%%%
\documentclass{ctexart}
\tableofcontents

\documentclass[12pt]{article}
\usepackage{cmap}
\usepackage{polyglossia}
\usepackage{geometry}
\usepackage{setspace}
\usepackage{enumitem}
\usepackage{graphicx} % Для графики
\usepackage{hyperref}
\usepackage{float}

% Настройки страницы
\geometry{a4paper, margin=2cm}
\setstretch{1.5}

% Счётчик для рисунков
\newcounter{figurecounter}
\setcounter{figurecounter}{0}

% Новая команда \myfigure{путь}{описание}{ширина}
\newcommand{\myfigure}[3]{%
  \stepcounter{figurecounter}%
  \begin{figure}[H]%
    \centering%
    \includegraphics[width=#3]{#1}%
    \caption{#2}%
    \label{fig:#1}%
  \end{figure}%
}

% Титульные данные
\title{报告 \\ 第5次实验作业 \\ 课程《聚类分析》}
\author{学生组号 22Б16-пу 沙拉巴林 М.C.\\指导教师 迪克 А.Г.}
\date{圣彼得堡,2025年}

\begin{document}

\begin{center}
    \textbf{圣彼得堡国立大学} \\
    应用数学与过程控制系 \\
    本科项目 \\
    <<大数据与分布式数字平台>> \\
    \vspace{1cm}
    \textbf{报告} \\
    关于第5次实验作业 \\
    课程《聚类分析》
\end{center}

\vspace{1cm}

\noindent
学生组号 22Б16-пу \hfill 沙拉巴林 М.C. \\
指导教师 \hfill 迪克 А.Г.

\vspace{2cm}

\begin{center}
圣彼得堡 \\
2025年
\end{center}

\newpage
\tableofcontents
\newpage

\section{工作目标}
开发一个系统来比较不同聚类算法,并根据给定指标对具有15个以上特征的数据集进行评估。

\section{任务描述(问题形式化)}

- 实现图形用户界面(GUI),用于设置聚类器参数
- 支持选择和配置多个聚类算法:CURE、Single Linkage、MaxMin Distance、ISODATA、FOREL
- 根据外部和内部指标评估聚类质量:
  - Rand Index
  - Jaccard Index
  - Fowlkes-Mallows Index
  - Phi Index
  - Compactness
  - Separation
- 实现特征选择方法:add\_method, del\_method, spa\_method, compactness, spread
- 执行聚类可视化(包括真实标签)
- 对比以下情况的聚类结果:
  - 不进行特征选择
  - 特征选择后
  - 去标识化数据上的结果

\section{程序说明}
\subsection*{输入数据:}
- 包含15+个特征的数据集(如果存在目标变量,则应位于最后列)

\subsection*{输出数据:}
- 各种图表
- 高质量指标
- 每种特征选择方法在不同指标下的比较

\section{算法结果分析}
使用以下数据集进行测试:

测试程序时使用的\texttt{iris.csv}数据集包含150个样本和4个特征。真实标签取自最后一列,并用于评估外部聚类质量指标。

\subsection*{程序设置:}
- 聚类数量:3
- 特征选择方法:add\_method
- 特征选择后的特征数量:2
- 质量指标:Rand Index, Jaccard Index, Fowlkes-Mallows Index, Phi Index, Compactness, Separation
- 数据标准化:启用

完成完整聚类流程(所有特征、特征选择后以及去标识化数据)后,得到以下结果:

- \textbf{图 1. 未进行特征选择}:Single linkage 和 ISODATA 表现良好。

\myfigure{2025-05-25-13-35-58.png}{图片说明}{\textwidth}

- \textbf{图 2. 使用特征选择}:FOREL 显示出最高的精度提升。

\myfigure{2025-05-25-13-36-10.png}{图片说明}{\textwidth}

- \textbf{图 3–7. 指标对比}:

\myfigure{2025-05-25-13-31-22.png.png}{测试结果}{\textwidth}
\myfigure{2025-05-25-13-38-26.png}{测试结果}{\textwidth}
\myfigure{2025-05-25-13-38-38.png}{测试结果}{\textwidth}
\myfigure{2025-05-25-13-38-48.png}{测试结果}{\textwidth}
\myfigure{2025-05-25-13-42-58.png}{测试结果}{\textwidth}
\myfigure{2025-05-25-13-43-08.png}{测试结果}{\textwidth}
  - Rand Index
  - Jaccard Index
  - Fowlkes
  - Phi Index
  - Compactness
  - Separation

\textbf{指标结论:}
- 分离度 → ISODATA
- 紧致性 → CURE
- 其他情况 → FOREL 或 Single Linkage

\textbf{最终流程如下:}
- 图 8. 报告生成流程图
\myfigure{2025-05-25-13-43-41.png}{终端图表}{\textwidth}
- 图 9. 图形界面视图
\myfigure{2025-05-25-13-46-53.png}{图形界面}{\textwidth}
- 图 10. 特征选择前的指标评估
\myfigure{2025-05-25-14-00-12.png}{指标评估}{\textwidth}
- 图 11. 去标识化数据上特征选择后的指标评估
\myfigure{2025-05-25-14-00-12.png.png}{特征选择后的指标评估}{\textwidth}
- 图 12. 结果对比
\myfigure{2025-05-25-14-02-27.png}{结果对比}{\textwidth}
- 图表 2. 获取数据的对比
\myfigure{2025-05-25-14-02-58.png.png}{获取数据的对比}{\textwidth}

Single Linkage 在聚类方面表现最佳。

\section{程序结构图}
- 图 13. 主程序结构图
\myfigure{2025-05-25-14-04-01.png}{结构图}{\textwidth}

\section{测试示例}
- 图 14. 图的解决方案示例
\myfigure{2025-05-25-14-04-33.png}{系统界面示例}{\textwidth}
- 图 15. 特征选择前的用户设置
\myfigure{2025-05-25-14-05-10.png}{特征选择前的用户设置}{\textwidth}

\section{总结}
在本次实验中,我们开发了一个用于比较各种聚类算法的系统,该系统包含用于参数设置和结果分析的图形界面。实现并测试了多种算法(CURE、Single Linkage、MaxMin Distance、ISODATA、FOREL),使用了原始数据、去标识化数据及特征选择后的数据。

分析表明,聚类质量取决于所选算法、特征选择方法和预处理方式。使用 $add_method$ 方法显著提高了外部和内部指标的表现,如 Rand Index、Jaccard Index、Fowlkes-Mallows Index、Phi Index、Compactness 和 Separation。最稳定的结果来自 CURE 和 ISODATA。

此外,还实现了聚类可视化功能,有助于直观地评估数据结构和算法效率。程序运行结果以图表和表格形式呈现,便于进行有效比较并得出合理结论。

\section{参考文献}
- Dorigo M., Di Caro G. \emph{"Ant Colony Optimization: A New Meta-Heuristic"}
- Tkinter 库: \url{https://docs.python.org/3/library/tkinter.html }
- Math 库: \url{https://docs.python.org/3/library/math.html }

\end{document}