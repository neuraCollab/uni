\documentclass[12pt,a4paper]{article}
\usepackage[utf8]{inputenc}
\usepackage[russian]{babel}
\usepackage{amsmath}
\usepackage{amsfonts}
\usepackage{amssymb}
\usepackage{graphicx}
\usepackage{listings}
\usepackage{xcolor}
\usepackage{hyperref}

\lstset{
    language=Python,
    basicstyle=\small\ttfamily,
    breaklines=true,
    frame=single,
    numbers=left,
    numberstyle=\tiny,
    numbersep=5pt,
    showstringspaces=false,
    keywordstyle=\color{blue},
    stringstyle=\color{red},
    commentstyle=\color{green!60!black},
    morekeywords={import,def,class,return,if,else,elif,for,while,in,True,False,None},
    extendedchars=true,
    inputencoding=utf8,
    literate={а}{{\cyra}}1
             {б}{{\cyrb}}1
             {в}{{\cyrv}}1
             {г}{{\cyrg}}1
             {д}{{\cyrd}}1
             {е}{{\cyre}}1
             {ё}{{\cyryo}}1
             {ж}{{\cyrzh}}1
             {з}{{\cyrz}}1
             {и}{{\cyri}}1
             {й}{{\cyrishrt}}1
             {к}{{\cyrk}}1
             {л}{{\cyrl}}1
             {м}{{\cyrm}}1
             {н}{{\cyrn}}1
             {о}{{\cyro}}1
             {п}{{\cyrp}}1
             {р}{{\cyrr}}1
             {с}{{\cyrs}}1
             {т}{{\cyrt}}1
             {у}{{\cyru}}1
             {ф}{{\cyrf}}1
             {х}{{\cyrh}}1
             {ц}{{\cyrc}}1
             {ч}{{\cyrch}}1
             {ш}{{\cyrsh}}1
             {щ}{{\cyrshch}}1
             {ъ}{{\cyrhrdsn}}1
             {ы}{{\cyrery}}1
             {ь}{{\cyrsftsn}}1
             {э}{{\cyrerev}}1
             {ю}{{\cyryu}}1
             {я}{{\cyrya}}1
             {А}{{\CYRA}}1
             {Б}{{\CYRB}}1
             {В}{{\CYRV}}1
             {Г}{{\CYRG}}1
             {Д}{{\CYRD}}1
             {Е}{{\CYRE}}1
             {Ё}{{\CYRYO}}1
             {Ж}{{\CYRZH}}1
             {З}{{\CYRZ}}1
             {И}{{\CYRI}}1
             {Й}{{\CYRISHRT}}1
             {К}{{\CYRK}}1
             {Л}{{\CYRL}}1
             {М}{{\CYRM}}1
             {Н}{{\CYRN}}1
             {О}{{\CYRO}}1
             {П}{{\CYRP}}1
             {Р}{{\CYRR}}1
             {С}{{\CYRS}}1
             {Т}{{\CYRT}}1
             {У}{{\CYRU}}1
             {Ф}{{\CYRF}}1
             {Х}{{\CYRH}}1
             {Ц}{{\CYRC}}1
             {Ч}{{\CYRCH}}1
             {Ш}{{\CYRSH}}1
             {Щ}{{\CYRSHCH}}1
             {Ъ}{{\CYRHRDSN}}1
             {Ы}{{\CYRERY}}1
             {Ь}{{\CYRSFTSN}}1
             {Э}{{\CYREREV}}1
             {Ю}{{\CYRYU}}1
             {Я}{{\CYRYA}}1
}

\title{Лабораторная работа №4\\Методы заполнения пропущенных значений в данных}
\author{Студент группы XXX}
\date{\today}

\begin{document}

\maketitle

\section{Введение}
В данной лабораторной работе рассматриваются различные методы заполнения пропущенных значений в наборах данных. Работа включает в себя реализацию и сравнение различных методов импутации, их визуализацию и оценку эффективности.

\section{Описание методов импутации}
В работе реализованы следующие методы заполнения пропущенных значений:

\subsection{Простыe методы}
\begin{itemize}
    \item Заполнение средним значением (mean)
    \item Заполнение медианой (median)
    \item Заполнение модой (mode)
    \item Заполнение предыдущим значением (ffill)
\end{itemize}

\subsection{Продвинутые методы}
\begin{itemize}
    \item Hot-deck импутация
    \item Линейная регрессия
    \item Стохастическая регрессия
    \item Сплайн-интерполяция
\end{itemize}

\section{Реализация}
\subsection{Основная структура проекта}
Проект состоит из следующих основных модулей:
\begin{itemize}
    \item \texttt{main.py} - основной файл для запуска анализа
    \item \texttt{data\_loading.py} - загрузка данных
    \item \texttt{data\_preprocessing.py} - предварительная обработка данных
    \item \texttt{imputation\_methods.py} - реализация методов импутации
    \item \texttt{evaluation.py} - оценка методов
    \item \texttt{visualization.py} - визуализация результатов
\end{itemize}

\subsection{Код реализации методов импутации}
\begin{lstlisting}[escapechar=!]
# Пример реализации метода линейной регрессии
def fill_missing(df, method="linear_regression", **kwargs):
    if method == "linear_regression":
        target_col = kwargs.get("target_col")
        feature_cols = kwargs.get("feature_cols")
        
        # !Заполнение пропусков в признаках!
        df_filled = df.copy()
        for col in feature_cols:
            if df_filled[col].isna().any():
                df_filled[col] = df_filled[col].fillna(df_filled[col].median())
        
        # !Обучение модели!
        known = df_filled[df_filled[target_col].notna()]
        unknown = df_filled[df_filled[target_col].isna()]
        
        model = LinearRegression()
        model.fit(known[feature_cols], known[target_col])
        predicted = model.predict(unknown[feature_cols])
        
        df_filled.loc[unknown.index, target_col] = predicted
        return df_filled
\end{lstlisting}

\section{Метрики оценки}
Для оценки эффективности методов импутации использовались следующие метрики:
\begin{itemize}
    \item Средняя относительная ошибка (MeanRelativeError\%) - показывает среднее отклонение предсказанных значений от истинных
    \item Ошибки в распределении данных - оценивают, насколько хорошо методы сохраняют статистические характеристики исходных данных:
    \begin{itemize}
        \item Ошибка в среднем значении
        \item Ошибка в стандартном отклонении
        \item Ошибка в квантилях распределения
    \end{itemize}
\end{itemize}

\section{Методология оценки}
Оценка методов проводилась следующим образом:
\begin{enumerate}
    \item Формирование датасета из полных наблюдений
    \item Внесение случайных пропусков в данные (3\%, 5\%, 10\%, 20\%, 30\%)
    \item Применение различных методов импутации
    \item Сравнение результатов с истинными значениями
    \item Оценка сохранения статистических характеристик данных
\end{enumerate}

Для каждого уровня пропусков проводилось 5 запусков для получения статистически значимых результатов.

\section{Результаты}
\subsection{Сравнение методов}
Тестирование проводилось на трех наборах данных разного размера:
\begin{itemize}
    \item Маленький набор данных (sm\_dataset)
    \item Средний набор данных (m\_dataset)
    \item Большой набор данных (lg\_dataset)
\end{itemize}

Для каждого уровня пропусков был определен лучший метод импутации на основе средней относительной ошибки. Результаты представлены в виде графика зависимости ошибки от процента пропусков для каждого метода.

\subsection{Выводы}
\begin{itemize}
    \item Простые методы (mean, median, mode) показывают хорошие результаты при малом проценте пропусков (до 5\%)
    \item При увеличении процента пропусков более эффективными становятся продвинутые методы (линейная регрессия, сплайн-интерполяция)
    \item Стохастическая регрессия показывает лучшие результаты в сохранении распределения данных
    \item Hot-deck импутация эффективна при наличии коррелированных признаков
\end{itemize}

\section{Заключение}
В ходе работы были реализованы и протестированы различные методы заполнения пропущенных значений. Каждый метод имеет свои преимущества и недостатки, и выбор конкретного метода зависит от специфики данных и требований задачи.

\end{document} 