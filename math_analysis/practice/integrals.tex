\documentclass[a4paper,12pt]{article}

\usepackage[utf8]{inputenc}
\usepackage[T1]{fontenc}
\usepackage[russian]{babel}
\usepackage{amsmath}
\usepackage{pgfplots}
\usepackage{animate}
\usepackage{graphicx}
\pgfplotsset{compat=1.18} % Укажите совместимую версию

\title{Название документа}
\author{Автор}
\date{\today}

\begin{document}

\maketitle

\section{Введение}
Закон, правило \textbf{f}, посредством которого каждому в
$a \in A$ сопоставляется единственный $b \in B$ называют \textbf{отображением}. Обычно
это записывают так: \[ b = f(a) \].

$f : A \to B$ -- отображение из \( A \) to \(B\)

$\forall c \in C: f(c) = f(c)$
$\vert x \vert \quad \text{или} \quad |x|$

$1^\text{sfds} \neq 2 \int x^2 \, dx \int_{0}^{1} x^2 \, dx \iint d21 \iiint dasd$

$\leftarrow \Rightarrow \Leftarrow \implies $
\section{Определения}

$a \leq b \infty \quad \lim_{x \to \infty} \quad \frac{a}{b} \sqrt[3]{21} \qquad f''(x )$

\textbf{3756.} $\int_{0}^{+\infty} \exp^{-\alpha x}\sin x \, dx$

\[
f(x) = 
\begin{cases}
    1, & \text{если} |x| \leq 1; \\
    0, & \text{если} |x| > 1 \\
    
\end{cases}
\]


\section{Заключение}
Ваш текст здесь.

\end{document}